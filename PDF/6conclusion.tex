% SPDX-FileCopyrightText: 2024 Lukas Zirpel <thesis+lukas@zirpel.de>
% SPDX-License-Identifier: GPL-3.0-only

\chapter{Conclusion}
\label{chap:conclusion}

Effective censorship evasion is crucial for maintaining access to information in restricted environments.
This thesis examines the overhead of various protocols designed to bypass censorship.
We present the results of experiments conducted under realistic network conditions.
This chapter synthesizes our findings, and identifies areas for future research to enhance these techniques.

While we are not able to fully evaluate all of the intended protocols, most notably ICMPTX and Tor pluggable transports, due to implementation challenges and the unsuitability of UDP traffic for the latter, we feel that our measurement setup and methodology provide a valuable foundation for future studies.
The rigorous approach we employ, combining virtualized prototyping with real-world hardware testing, allows for reproducible experiment environments and controlled network impairment simulation.
This framework can be readily adapted to evaluate a broader range of censorship evasion techniques and explore more complex network scenarios.

As such, this framework is the core contribution of this thesis. The evaluation framework can be found at \href{https://github.com/Luflosi/Bachelor-Thesis}{https://github.com/Luflosi/Bachelor-Thesis} and \href{https://sr.ht/~luflosi/Bachelor-Thesis/}{https://sr.ht/\textasciitilde{}luflosi/Bachelor-Thesis/}.

RAM usage of all the protocols we evaluated was quite low.
It was low enough to comfortably run these protocols on embedded devices or old mobile phones.


\section{Future Work}
Building upon the findings and limitations of our research (\Cref{chap:discussion}), this section outlines several promising avenues for future investigation.
These directions aim to expand the scope of our analysis, improve the accuracy of our measurements, and provide more comprehensive insights into the effectiveness of censorship circumvention techniques.

\noindent\textbf{Improved Experimental Setup:}
While our setup allows for controlled network impairment simulation, limitations were encountered in accurately measuring certain metrics, such as protocol-induced packet loss.
Future work could focus on developing an improved experimental (\Cref{fig:optimal_network_schematic}) setup that addresses these limitations.

\noindent\textbf{Expanded Protocol Evaluation:}
The current study evaluates a selection of commonly used censorship circumvention protocols.
Future research could expand this evaluation to include a wider range of protocols, such as I2P or Shadowsocks.
Alternative implementations of e.g., the IP over ICMP concept (e.g., Hans \cite{hans}) may also be worth looking at.
This would provide a more comprehensive understanding of the performance landscape and identify potential strengths and weaknesses of different approaches.

\noindent\textbf{Bandwidth Limit:}
Although we vary parameters such as latency and packet loss, we do not reduce the available bandwidth below what the hardware could provide.
Future research could explore introducing an artificial bottleneck to simulate lower-bandwidth conditions.

\noindent\textbf{Connection Oriented Protocols:}
Our experiments primarily utilize unidirectional UDP traffic as the foundation for our analysis.
However, future research could benefit from exploring the impact of connection-oriented protocols such as TCP and QUIC.
These protocols are widely used in real-world Internet applications, which makes their inclusion potentially more relevant for practical insights.
It is important to note that most TCP implementations are sensitive to packet reordering, a factor that could influence performance evaluations.
Additionally, assessing the role of congestion control mechanisms within these protocols could provide valuable perspectives on how they operate under different network conditions.

\noindent\textbf{IPv6:}
In our experiments, we exclusively utilize IPv4, both inside and outside of the tunnels.
ICMPTX lacks support for ICMPv6, which means it does not work over IPv6 outside of the tunnel.
Additionally, iodine does not support IPv6 within the tunnel.
These limitations make it more challenging to compare all protocols in an IPv6 environment.
Nonetheless, future research could still explore this aspect further.

\noindent\textbf{Constant MTU Instead of Constant Payload Size:}
In our experiments, we maintain a consistent payload size within the tunnel protocol for different protocols.
Keeping the packet size outside the tunnel constant instead, simulates a fixed Maximum Transmission Unit (MTU).
This method provides an alternative way to conduct such experiments.
We suggest that future research could adopt this latter approach—controlling external packet size to simulate a fixed MTU—as a potential improvement over the current methodology.
This alternative method offers a different perspective on how network parameters can be managed in experimental settings, potentially leading to more accurate or insightful results.

\noindent\textbf{Dynamic Network Conditions:}
This research primarily focuses on evaluating protocols under static network impairment conditions.
Future studies could investigate the performance of these protocols under dynamic network conditions, where impairments change over time.
Simulating transitions between different network states (e.g., from low latency to high latency, or from low packet loss to high packet loss) would provide valuable insights into the adaptability and resilience of the protocols in real-world scenarios.
This could involve developing test scenarios that simulate gradual or abrupt changes in network parameters, mimicking the fluctuating conditions often encountered in mobile or wireless networks.

\noindent\textbf{Security Analysis:}
While this study focuses on performance, security is a critical aspect of censorship circumvention.
Future work could include a  in-depth security analysis of the evaluated protocols, examining their resistance to various attack vectors and potential vulnerabilities.
This could involve a code review.

\noindent\textbf{User Studies:}
Ultimately, the effectiveness of censorship circumvention tools depends on their usability and user experience.
Future research could involve conducting user studies to evaluate the practicality and user-friendliness of the evaluated protocols in real-world settings.
This could provide valuable feedback for improving the design and implementation of these tools.


%%% Local Variables:
%%% TeX-master: "thesis"
%%% End:
