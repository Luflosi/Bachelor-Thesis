% SPDX-FileCopyrightText: 2024 Lukas Zirpel <thesis+lukas@zirpel.de>
% SPDX-License-Identifier: GPL-3.0-only

\chapter{Conclusion}
\label{chap:conclusion}

\todo{kurz und bündig Forschungsfragen beantworten}
Effective censorship evasion is crucial for maintaining access to information in restricted environments.
This thesis examines the overhead of various protocols designed to bypass censorship.
We present the results of experiments conducted under realistic network conditions.
This chapter synthesizes our findings, and identifies areas for future research to enhance these techniques.


while we were not able to ...

we feel that measurement setup ...

\section{Future Work}
more protocols

better setup

use IPv6

Test changing network conditions (good to bad or vice versa)

vary maximum bandwidth

To gain deeper insights into real-world network behavior, we should also test using connection-oriented protocols like TCP or QUIC to evaluate the impact of congestion control mechanisms.

In our testing, we kept the size of the payload inside the tunnel protocol constant.
Keeping the packet size outside of the tunnel constant simulates a fixed MTU much more accurately and is thus ...

test obfs4

test the tor transport? \todo{read some more about this transport protocol}

Try IP over DNS implementations other than ICMPTX, such as \href{https://code.gerade.org/hans/}{Hans}.


%%% Local Variables:
%%% TeX-master: "thesis"
%%% End:
