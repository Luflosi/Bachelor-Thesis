% SPDX-FileCopyrightText: 2024 Lukas Zirpel <thesis+lukas@zirpel.de>
% SPDX-License-Identifier: GPL-3.0-only

\chapter{Background and Related Work}
\label{chap:background}

\todo{There must be some analysis on WireGuard}
\todo{Split related work into separate chapter?}

There are a number of \textit{existing network simulators} \cite{network-simulators-list} for simulating complex network setups with routers, switches, etc..
Many of them focus on simulating mesh networks for testing routing protocols, for example \textit{Mesh Network Lab} \cite{meshnet-lab}.
Our network is relatively simple, while the complexity lies in the configuration of the hosts.
This is why we chose the NixOS testing framework for our first prototype.
It allows declararing the exact configuration of the hosts and the virtual network in code.
All dependencies are pinned in a lock file and no outside dependencies are used so everything should still be reproducible in a decade or more without having to e.g. hunt for ancient Ubuntu images.

\section{Network data transmission and MTU}
\textit{MTU} \cite{wiki:Maximum_transmission_unit}:
When the MTU inside of a tunnel is so large, that the overhead of the tunnel protocol would make the packet too big to be transported over the internet
Needs to be dealt with in some way. Usually by either dropping the packet and/or communicating the error to the application (\textit{Path MTU discovery} \cite{wiki:Path_MTU_Discovery}) or by fragmenting the packet. Since fragmentation has a relatively large performance overhead, this is usually avoided in practice. For this reason, we avoid fragmentation in this research by carefully choosing the MTU inside of the tunnels.

\section{Censorship}
\subsection{Preventing content access}
generell censorship
content access
net neutrality
ist technisch ähnlich
wie läuft das ab, beispiel:
There are numerous ways to do censorship online, so the following is structured in a cat-and-mouse game between two fictional parties, a censor Bob and a person Alice:
Bob wants to censor what Alice can see online.
Bob operates an ISP and Alice is connected to it via DSL. Alice gets her IP address from Bob's DHCP server.
The DHCP server also advertises a DNS resolver to Alice, which she gladly uses.
Now Bob decides that he does not want Alice to be able to access the website example.org.
He wants to keep the collateral damage low by trying to not block any other websites.
As a first step, he modifies his DNS resolver.
Instead of replying to a query directly with the correct reply, it first checks if the DNS query asks for example.org.
If so, it replies with a different answer.
If the query was for example for the AAAA record of example.org, Bob could make his DNS server reply with the IPv6 address ::1, which is not the true address of example.org, thereby preventing Alice from accessing it.
To circumvent the censorship, alice can simply choose a different DNS resolver, which does not lie about its responses.
Or Alice could even run her own recursive DNS resolver.

\todo{continue}

wie gesehen, gibt es viele Möglichkeiten der evasion:

\subsection{Censorship evasion}

\section{Network Throughput Performance Measurements}
Network throughput is a measure of how much data can be transmitted over a given communication channel per time.
Every communication protocol has a certain overhead, including tunnel protocols, reducing the maximum possible useful network throughput called goodput.
Goodput can be influenced by many factors including every piece of hardware in the chain from one end of the communication to the other, retransmissions due to packet loss, protocol overhead and many more.

related work: \textit{Promises and Potential of BBRv3} \cite{Promises-and-Potential-of-BBRv3}\todo{Also link \href{https://pam2024.cs.northwestern.edu/pdfs/paper-59.pdf}{the PDF} somehow}

\section{Reproducibility in Network Measurement}
Achieving reproducibility for any performance measurement is tricky. Both hard- and software need to be reproduced as closely as possible.
