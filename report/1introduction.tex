% SPDX-FileCopyrightText: 2024 Lukas Zirpel <thesis+lukas@zirpel.de>
% SPDX-License-Identifier: GPL-3.0-only

\chapter{Introduction}


Many places in the world encounter internet censorship. To try to get around the censorship, numerous different techniques, some very creative, were developed to obfuscate data and either hide it from or make it seem like innocent data to the censors. These censorship circumvention protocols are likely developed in laboratory conditions with ideal networks and plentiful computational resources. These are not the same conditions encountered in the real world, where networks may have a high latency and low throughput. Running these protocols on phones or other mobile devices also constrains the available CPU and RAM resources. For this reason, we aim to analyze the behaviour of different censorship circumvention protocols under these conditions.
