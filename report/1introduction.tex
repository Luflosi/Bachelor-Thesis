% SPDX-FileCopyrightText: 2024 Lukas Zirpel <thesis+lukas@zirpel.de>
% SPDX-License-Identifier: GPL-3.0-only

\chapter{Introduction}


Many places in the world encounter Internet censorship.
To try to get around the censorship, numerous different techniques, some very creative, were developed.
Depending on the threat model, the goal can be to either "just" circumvent the censorship or to also hide the fact that circumvention took place from the censors.
Steganography \cite{wiki:Steganography} techniques \cite{wiki:List_of_steganography_techniques} can help with the latter goal.
Internet censorship doesn't happen equally everywhere.
It is often employed to restrict access of a population to sensitive information and, as such, is much more likely to be present on types of Internet connections used by many people, as opposed to e.g., data centers.
These types of Internet connections are optimized for low cost and include access technologies such as cellular networks, DSL, and WLAN.
People are likely using portable or low-power devices such as laptops and phones, as opposed to servers.

Censored Internet connections are also commonly found in public spaces like coffee shops, hotels, and city-wide WiFi networks, where users rely on shared, often unencrypted, and bandwidth-limited connections.


The Open Observatory of Network Interference (OONI) project measures Internet censorship \cite{OONI}.
Internet censorship in the real world is constantly evolving and can be done in many different ways.
What follows is a small selection of the most common ways censorship is achieved on a technical level and for a service outside of the jurisdiction or control of the censor:
\begin{itemize}
	\item \noindent\textbf{DNS Blocking:} Preventing resolution of specific domain names by returning NXDOMAIN or other error responses.
	\item \noindent\textbf{DNS Spoofing/Hijacking:} Redirecting requests for specific domains to different IP addresses, often a "block page".
	\item \noindent\textbf{IP/Connection Blocking:} Blocking access to specific IP addresses or ports associated with censored content or based on real-time analysis of traffic.
	\item \noindent\textbf{TCP Reset Injection:} Injecting TCP RST packets to terminate connections to blocked servers.
	\item \noindent\textbf{Blocking or throttling specific protocols} (e.g., VoIP, peer-to-peer file sharing).
	\item \noindent\textbf{Traffic Shaping/Throttling:} Slowing down or prioritizing certain types of traffic to make access to censored content difficult or unusable.
\end{itemize}
All of these censorship techniques have different tradeoffs in terms of false positive rate, false negative rate, ease of circumvention, and computational cost.

These types of censorship can be circumvented in a variety of ways.
Fake TCP RST packets sent by unsophisticated firewalls for example can often be differentiated from the real ones and then simply ignored \cite{GoodbyeDPI-passive}.
More sophisticated firewalls can be circumvented by tunneling.

Tunneling is a method used to evade censorship by encapsulating data within another protocol, making it difficult for firewalls to detect restricted content.
This technique effectively hides the true nature of the traffic, allowing it to pass through filters undetected.
Tunneling protocols often employ encryption to maximize their effect.

Running these protocols on phones or other mobile devices also constrains the available CPU and RAM resources.
However, these censorship circumvention protocols are likely developed in laboratory conditions with ideal networks and plentiful computational resources.
These are not the same conditions encountered in the real world, where networks may have a high latency, high packet loss and low throughput.

For this reason, we aim to analyze the behaviour of different censorship circumvention protocols under these conditions.
Specifically, we ask the research question:
\emph{``How do censorship circumvention protocols react to bad network conditions?''}
This research question breaks up into the following sub-questions:
\begin{enumerate}
  \item How much packet size overhead does each protocol add?
  \item How much does the MTU decrease by using each protocol?
  \item How much additional latency does each protocol add?
  \item Do any protocols introduce additional packet loss?
  \item How much processing power and RAM does each protocol consume/require?
\end{enumerate}

We answer these research questions using a simulation environment.
Accurately simulating real-world conditions is challenging.
We use tc-netem \cite{man8:tc-netem}, Linux's network emulator, as an approximation of the real world.
We first use the NixOS testing framework to simulate our test setup using VMs for experimentation, and then run the setup on real hardware to make sure that our results are not influenced by the virtualisation.

Since censorship circumvention protocols operate as network tunnels, transporting the actual payload within them, we also examine their performance for pathological behavior, specifically excessive packet loss or latency inside the tunnel relative to the outside network, under unfavorable conditions


\section{Structure of Thesis}
The remainder of this thesis is structured as follows:\\
In \Cref{chap:background}, we summarize previous research on the topic and introduce background information and terms.
Subsequently, in \Cref{chap:methodology}, we describe our test setup and how we measure things.
Next, in \Cref{chap:results}, we show the results of our measurements and compare them to each other.
In \Cref{chap:discussion}, we answer our research questions and discuss interesting and surprising findings.
We also reflect on the limitations of our testing methodology, and on areas where we could have done better.
Finally, in \Cref{chap:conclusion}, we briefly summarize our research and discuss future work.


%%% Local Variables:
%%% TeX-master: "thesis"
%%% End:
