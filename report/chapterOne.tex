% SPDX-FileCopyrightText: 2024 Lukas Zirpel <thesis+lukas@zirpel.de>
% SPDX-License-Identifier: GPL-3.0-only

\chapter{Introduction}

\setCoqFilename{filename}

\begin{theorem}[Test][theoremname]
  This is a great theorem.
\end{theorem}

We refer to Theorem \ref{coq:theoremname}.

In an attempt to save space...
The .pcap files only compress well if the data is not encrypted or otherwise scrambled.
The files could either be explicitly compressed with a compression tool like zstd or the compression could be done at the filesystem level, e.g. with ZFS.


Since we're only interested in measuring what happens during the actual test and not in the connection setup and teardown of iperf3, we use heuristics
The connection setup and teardown of iperf3 should not be part of the analysis, hence a heuristic is employed to ignore this part of the packet capture. To find the start of the test, we find the first packet which is larger than a threshhold. To find the end of the test, we find the last packet which is larger than a threshhold and also ignore all packets that are sent after the duration of the test is over.

%%% Local Variables:
%%% TeX-master: "thesis"
%%% End:
